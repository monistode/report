\documentclass[conference]{IEEEtran}
\IEEEoverridecommandlockouts
% The preceding line is only needed to identify funding in the first footnote. If that is unneeded, please comment it out.
\usepackage{cite}
\usepackage{amsmath,amssymb,amsfonts}
\usepackage{algorithmic}
\usepackage{graphicx}
\usepackage{textcomp}
\usepackage{xcolor}
\usepackage{hyperref}
\def\BibTeX{{\rm B\kern-.05em{\sc i\kern-.025em b}\kern-.08em
		T\kern-.1667em\lower.7ex\hbox{E}\kern-.125emX}}
\begin{document}

\title{Implementation of Educational Instruction Set Architectures}

\author{\IEEEauthorblockN{Bohdan Opyr}
	\IEEEauthorblockA{\textit{Faculty of Applied Sciences} \\
		\textit{Ukrainian Catholic University}\\
		L'viv, Ukraine \\
		opyr.pn@ucu.edu.ua}
	\and
	\IEEEauthorblockN{Petro Mozil}
	\IEEEauthorblockA{\textit{Faculty of Applied Sciences} \\
		\textit{Ukrainian Catholic University}\\
		L'viv, Ukraine \\
		mozil.pn@ucu.edu.ua}
	\and
	\IEEEauthorblockN{Radomyr Husiev}
	\IEEEauthorblockA{\textit{Faculty of Applied Sciences} \\
		\textit{Ukrainian Catholic University}\\
		L'viv, Ukraine \\
		husiev.pn@ucu.edu.ua}
	\and
	\IEEEauthorblockN{Roman Zaletsky}
	\IEEEauthorblockA{\textit{Faculty of Applied Sciences} \\
		\textit{Ukrainian Catholic University}\\
		L'viv, Ukraine \\
		zaletsky.pn@ucu.edu.ua}
}

\maketitle

\begin{abstract}
	This document contains a description of the implementation of the
	four instruction set architectures, as well as the development environment created
	for facilitating the learning of the differences of these architectures.
	This includes an assembler, linker and a disassebmler for these ISAs, an emulator
	and a the processor implementations on a FPGA devboard, specifically DE10-Nano rev. C,
	alongside a linux image for facilitating the writing and uploading the compiled code.

	The linux image is debian 10 buster with linux 6.1.38 and the assembler is
	customizeable -- one can write a YAML file to describe practicallly any ISA.
\end{abstract}

\begin{IEEEkeywords}
	SystemVerilog, Field Programmable Gate Array, Instruction Set Architecture, Assembler, Linker, Emulator
\end{IEEEkeywords}

\section{Introduction}
An ISA (instruction sert architecture) is an abstract model of a computer.
In general, an ISA defines the supported instructions, data types, registers,
the hardware support for managing main memory, fundamental features
(such as the memory consistency, addressing modes, virtual memory),
and the input/output model of a family of implementations of the ISA. \cite{b7}

Modern processors are complicated. That makes them unsuitable for educational purposes, and
especially for teaching the difference between different types of ISA's, since even the
most noticeable differences between different types of processors are hidden behind the afforementioned complexity.
Even RISC-V is a very extensive and complex ISA, which has a lot of concepts that are fairly
difficult to understand at a first glance.

For this reason, the four instruction set architectures we implemented were created. \cite{b1}
The ISAs themselves are not meant to be optimal or productive -- they were created to show
the basics of instruction decoding and the difference between a stack, accumulator, RISC and CISC architectures.

Such an educational project would also greatly benefit from an assembler / disassebmler and an emulator,
apart from the processor itself. Our objective is to create not only a processor, but also an environment
that is easy to develop in. The assembler should return clear errors if there are any,
and so should the emulator. It should be easy to upload the programs to the processor,
and there sould be a way to execute instructions step-by-step.

We created the processors that implement the four ISAs and an environment to develop
programs for them / test these programs.

The code and documentation for the project can be found on Github at monistode \cite{b9}

\section {ISA description}
The four ISAs are as such:
\begin{itemize}
	\item A stack processor ISA.
	\item An accumulator processor ISA.
	\item A RISC ISA.
	\item A CISC ISA.
\end{itemize}

The four architectures have only the most basic commands:
\begin{itemize}
	\item Arithmetic (add, compare, multiply, subtract and divide)
	\item Basic bit manipulation opeations (bitwise and, or, xor and not, and the test command)
	\item Simple load / store commands
	\item Very simple IO: UART connection and multiple GPIO pins
\end{itemize}

Each architecture has its own unique features, e. g. the CISC one has a couple of SIMD instructions.

\subsection {Stack ISA \cite{b4}}
The stack ISA is a harvard architecture.
It has an address space of size 65536 bytes, the byte is 6 bits,
and the machine word is 2 bytes. The addressing modes are:
\begin{itemize}
	\item Register (from the top of the register stack)
	\item Immediate
	\item Memory location
\end{itemize}

The instruction size is 6 to 18 bits (2 byte immediate). The first bit signifies whether there is an immediate present.

There are two stacks:
\begin{itemize}
	\item Register stack - 16 bits per item, stores the data for operations.
	      Starts at the 256th byte, grows upward.
	\item Memory stack - 16 bits per item, stores the function return addresses.
	      Starts at the 1024th byre, grows downward.
\end{itemize}

Stack underflow nor collisions with the register stack not handled


\newpage
There are four registers:
\begin{itemize}
	\item PC - 16 bits - program counter, points to a 6-bit byte in the text section. Is always the next instruction to be executed, similar to x86
	\item FR - 4-16 bits - flag register with the least significant bits representing CF ZF OF SF; (up to 16 bits because it has to fit on the stack)
	\item TOS - 15 or 16 bits - initial value 256 (assuming 16 bits) - points to two bytes in the data section - the top of the register stack; grows upward (a push increments)
	\item SP - 15 or 16 bits - initial value 1024 (assuming 16 bits) - points to two bytes in the data section - the top of the memory stack; grows downward (a push subtracts)
\end{itemize}

\subsection {Accumulator ISA \cite{b6}}
The accumulator ISA is a von Neumann architecture.
It has an address space of size 65536 bytes, the byte is 8 bits,
and the machine word is 16 bits. The addressing modes are:
\begin{itemize}
	\item Register (from the top of the register stack)
	\item Immediate
	\item Memory location
\end{itemize}

The instruction size is 8 to 24 bits (the immediate is 2 bits). The first bit signifies whether there is an immediate present.

There is one stack - the memory stack. It starts at 1024 byte and grows downwards.

There are four registers:
\begin{itemize}
	\item PC - 16 bits - program counter, points to a 6-bit byte in the text section. Is always the next instruction to be executed, similar to x86
	\item FR - 16 bits - flag register with the least significant bits representing CF ZF OF SF; (up to 16 bits because it has to fit on the stack)
	\item SP - 16 bits - initial value 1024 (assuming 16 bits) - points to two bytes in the data section - the top of the memory stack; grows downward (a push subtracts)
	\item IR1 - 16 bits - index register (used for indexing in arrays)
	\item IR2 - 16 bits - index register (used for indexing in arrays)
	\item ACC - 16 bits - the accumulator
\end{itemize}

IR1 and IR2 can only be incremented / decremented and stored/read from.

\subsection {RISC ISA}
The RISC ISA is a von Neumann architecture.
It has an address space of size 65536 bytes, the byte is 8 bits,
and the machine word is 16 bits. The addressing modes are:
\begin{itemize}
	\item Register (from the top of the register stack)
	\item Immediate
	\item Memory location
\end{itemize}

The instruction size is 8 to 16 bits. The first bit signifies whether there is an immediate present.

There is one stack - the memory stack. It starts at 1024 byte and grows downwards.

There are four registers:
\begin{itemize}
	\item PC - 16 bits - program counter, points to a 6-bit byte in the text section. Is always the next instruction to be executed, similar to x86
	\item FR - 16 bits - flag register with the least significant bits representing CF ZF OF SF; (up to 16 bits because it has to fit on the stack)
	\item SP - 16 bits - initial value 1024 (assuming 16 bits) - points to two bytes in the data section - the top of the memory stack; grows downward (a push subtracts)
	\item R00, R01, R02, R03 - 16 bits each, with L and H bits each (they, however, cannot be accesed easily)
\end{itemize}

\subsection {CISC ISA \cite{b5}}
The CISC ISA is a von Neumann architecture.
It has an address space of size 65536 bytes, the byte is 8 bits,
and the machine word is 16 bits. The addressing modes are:
\begin{itemize}
	\item Register (from the top of the register stack)
	\item Immediate
	\item Memory location
\end{itemize}

The instruction size is 8 to 64 bits. The first bit signifies whether there is an immediate present.

There is one stack - the memory stack. It starts at 1024 byte and grows downwards.

There are four registers:
\begin{itemize}
	\item PC - 16 bits - program counter, points to a 6-bit byte in the text section. Is always the next instruction to be executed, similar to x86
	\item FR - 16 bits - flag register with the least significant bits representing CF ZF OF SF; (up to 16 bits because it has to fit on the stack)
	\item SP - 16 bits - initial value 1024 (assuming 16 bits) - points to two bytes in the data section - the top of the memory stack; grows downward (a push subtracts)
	\item R00, R01, R02, R03 - 16 bits each, with L and H bits each (they, however, cannot be accesed easily)
\end{itemize}

There are SIMD instructions - they store the data into R00 through R03 and perform them as such:
\begin{itemize}
	\item R00L | R02L
	\item R00H | R02H
	\item R01L | R03L
	\item R01H | R03H
\end{itemize}

There is SIMD addition, subtraction division and multiplication, and also load and store into.
The load gets 8 bytes in a row, and the store stores 8 bytes in a row.

\newpage
\section {Processor implementation}

The processors runs on FPGA, specififcally the DE10-Nano devboard. The devboard
also has a linux image running in parallel to the processor. The FPGA and the linux image
share 512 megabytes of RAM. The entry point for the program is 0.
The entry point from linux is at address 0x20000000. The processor cannot access any of the
512 megabytes before 0x20000000, and it can only access megabytes from 512 to 515.

We created the processors in SystemVerilog, and made them fairly easy to update and customize.

For memory we used Avalom memory map interface \cite{b8} - it is an interface we used for
simplifying RAM and UART IO.

We also used the quartus IDE for development.

Every processor has UART IO on such ports:
\begin{itemize}
	\item RX - GPIO0 pin 0
	\item TX - GPIO0 pin 1
	\item GND - any ground on the DE10-Nano
\end{itemize}

The GPIO pins on the CPU are GPIO0 pins 2 through 36.

The debug mode (setp-by-step execution) mode is turned on by SW3.
The stepping is done by pressing KEY1, and unhalting is done by pressing KEY1.

When SW1 is on, the execution is stopped and pressing KEY0 resets the processor.


\section {Assembler, Disassembler and Linker \cite{b2}}
Since performance isn't a key concern, but instead simplicity, the assembler and
linker are written in python. The assembler is a single pass assembler based on
a simple recursive descent parser. The linker takes in a list of object files
and outputs a single executable file, after applying the necessary relocations.
An object file can hold text and data sections, the symbol table and the
relocation table. The executable holds segments, each of which can be marked as
executable or non-executable, readable or non-readable, writable or non-writable
and special or non-special. It is placed into a specific address in the address
space. The disassembler simply crawls each text section command by command,
returning the disassembled code. The linker can also output the executable as two
raw files, each of which is a representation of the raw address space of the
executable.

\section {Program upload}
The uploading of the program is simply done by writing 65536 bytes of the given program
to location 0x20000000 on linux. This pattern in widely used by SoC FPGA developers:
We open "/dev/mem" and mmap 64Kb of memory at address 0x20000000. After that we write to
that memory and unmap it.

\section {Conclusion}
We designed a workflow for writing, assembling and executing of programs written for the four educational ISAs.
The implementations are open-source and the customization of them is encouraged.

There are four repositories containing the quartus projects and three repos containing the assembler, linker and emulator for the
project.

We also updated the documentation and the accumulator ISA.


\begin{thebibliography}{00}
	\bibitem{b1}
	Oleg Farenyuk, Darian Omelkina. Code for the original assembler and simulator.
	\url{https://github.com/ucu-computer-science/Hardware-Simulator-and-Assembler/tree/master}

	\bibitem{b2}
	Monistode. Code for the assembler and linker.
	\url{https://github.com/monistode/binutils}

	\bibitem{b3}
	Monistode. Code for ISA documentations
	\url{https://github.com/monistode/ISA-docs}

	\bibitem{b4}
	Monistode. Code for the stack ISA
	\url{https://github.com/monistode/ISA_stack}

	\bibitem{b5}
	Monistode. Code for the CISC ISA
	\url{https://github.com/monistode/ISA_cisc}

	\bibitem{b6}
	Monistode. Code for the accumulator ISA
	\url{https://github.com/monistode/ISA_accum}

	\bibitem{b7}
	Wikipedia. Definition of ISA
	\url{https://en.wikipedia.org/wiki/Instruction_set_architecture}

	\bibitem{b8}
	Avalon interface specification
	\url{https://cdrdv2-public.intel.com/667068/mnl_avalon_spec-683091-667068.pdf}

	\bibitem{b9}
	The monistode organization
	\url{https://github.com/monistode}
\end{thebibliography}

\end{document}
